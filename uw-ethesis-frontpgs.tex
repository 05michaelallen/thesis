% T I T L E   P A G E
% -------------------
% Last updated Nov 1, 2016, by Stephen Carr, IST-Client Services
% The title page is counted as page `i' but we need to suppress the
% page number.  We also don't want any headers or footers.
\pagestyle{empty}
\pagenumbering{roman}

% The contents of the title page are specified in the "titlepage"
% environment.
\begin{titlepage}
        \begin{center}
        \vspace*{1.0cm}

        \Huge
        {\bf A method for hemispherical ground based remote sensing of urban surface temperatures}
        % Hemispherical Remote Sensing of Urban Surface Temperature
        %Hemispherical Remote Sensing of Urban Thermal Climates
        %A Method to Assess the Surface Urban Heat Island Effect Using Hemispherical Radiometric Surface Temperatures

        \vspace*{1.0cm}

        \normalsize
        by \\

        \vspace*{1.0cm}

        \Large
        Michael A. Allen \\
        
		%\vspace*{1.0cm}
        \vspace*{3.0cm}

        \normalsize
        A thesis \\
        presented to the University of Western Ontario\\ 
        in fulfillment of the \\
        thesis requirement for the degree of \\
        Master of Science \\

        \vspace*{2.0cm}

        London, Ontario, Canada, 2017 \\

        \vspace*{1.0cm}

        \copyright\ Michael A. Allen 2017 \\
        \end{center}
\end{titlepage}

% The rest of the front pages should contain no headers and be numbered using Roman numerals starting with `ii'
\pagestyle{plain}
\setcounter{page}{2}

\cleardoublepage % Ends the current page and causes all figures and tables that have so far appeared in the input to be printed.
% In a two-sided printing style, it also makes the next page a right-hand (odd-numbered) page, producing a blank page if necessary.


% A B S T R A C T
% ---------------

\begin{center}\textbf{Abstract}\end{center}

This thesis presents a method for deriving time-continuous urban surface temperature and heat island assessments from hemispherical ground-based measurements of upwelling thermal radiation. The method, developed to overcome geometric and temporal biases inherent in traditional thermal remote sensing of urban surface climates, uses a sensor view model in conjunction with a radiative transfer code to derive atmospherically corrected, hemispherical radiometric urban surface temperatures. These are used to derive two long-term climatologies of surface urban heat island (sUHI) magnitudes for Basel, Switzerland and Vancouver, Canada. sUHI development shows significant variation based on time-of-day, season, and ambient and synoptic conditions. Results also show large differences in remote sensed sUHI from hemispherical, nadir and complete representations of the urban surface, with a nadir view overestimating seasonal sUHI\textsubscript{max} from a complete view by nearly a factor of two. In contrast, a hemispherical view provides significantly more representative, time-continuous urban surface temperature and sUHI analysis.

\cleardoublepage

% A C K N O W L E D G E M E N T S
% -------------------------------

\begin{center}\textbf{Acknowledgments}\end{center}

To be filled in at a later date.
\cleardoublepage

% T A B L E   O F   C O N T E N T S
% ---------------------------------
\renewcommand\contentsname{Table of Contents}
\tableofcontents
\cleardoublepage
\phantomsection    % allows hyperref to link to the correct page

% L I S T   O F   T A B L E S
% ---------------------------
\addcontentsline{toc}{chapter}{List of Tables}
\listoftables
\cleardoublepage
\phantomsection		% allows hyperref to link to the correct page

% L I S T   O F   F I G U R E S
% -----------------------------
\addcontentsline{toc}{chapter}{List of Figures}
\listoffigures
\cleardoublepage
\phantomsection		% allows hyperref to link to the correct page

% GLOSSARIES (Lists of definitions, abbreviations, symbols, etc. provided by the glossaries-extra package)
% -----------------------------
\chapter*{List of Symbols}
\addcontentsline{toc}{chapter}{List of Symbols}

\captionsetup[table]{list=no}

\begin{table}[H]
	\centering
	\begin{tabular}{p{1.5cm}p{2cm}p{10cm}}
		Roman &&\\
		\toprule
		
		Symbol & Unit & Property \\
		\midrule
		$C$ & \si{\joule} & heat capacity \\
		FOV & \si{\deg} & sensor field of view \\
		\textit{H} & \si{\meter} & building width \\
		$Q*$ & \si{\watt\per\square\meter} & net radiation flux density \\
		$\Delta Q_A$ & \si{\watt\per\square\meter} & net advective heat flux density \\
		$\Delta Q_G$ & \si{\watt\per\square\meter} & net heat storage flux density \\
		$Q_E$ & \si{\watt\per\square\meter} & latent heat flux density \\
		$Q_F$ & \si{\watt\per\square\meter} & anthropogenic heat flux density \\
		$Q_H$ & \si{\watt\per\square\meter} & sensible heat flux density \\
		$t$ & \si{\minute} & time \\
		T\textsubscript{air} & \si{\kelvin} & air temperature \\
		T\textsubscript{surf} & \si{\kelvin} & surface temperature \\
		T\textsubscript{comp} & \si{\kelvin} & complete urban surface temperature\\
		T\textsubscript{plan} & \si{\kelvin} & urban surface temperature viewed from a sensor in the nadir\\
		T\textsubscript{hem} & \si{\kelvin} & urban surface temperature viewed from a hemispherical remote sensor\\
		aUHI & \si{\kelvin} & air temperature urban heat island effect \\
		clUHI & \si{\kelvin} & canopy layer air temperature urban heat island effect \\
		sUHI & \si{\kelvin} & surface temperature urban heat island effect \\
		$V_B$ & - & effects from background climate on observation of a given meteorological variable \\
		$V_H$ & - & effects from the urban environment on observation of a given meteorological variable \\
		$V_L$ & - & effects from local topography on observation of a given meteorological variable \\
		$V_M$ & - & observed meteorological variable \\
		$V_{M, U} $ & - & observed meteorological variable (urban) \\
		$V_{M  R}$ & - & observed meteorological variable (rural) \\
		$z$ & \si{\meter} & layer thickness (air, surface, or subsurface)\\
		\bottomrule
	\end{tabular} 
\end{table}

\begin{table}[H]
	\centering
	\begin{tabular}{p{1.5cm}p{2cm}p{10cm}}
		Roman &&\\
		\toprule
		
		Symbol & Unit & Property \\
		\midrule
		$C$ & \si{\joule} & heat capacity \\
		FOV & \si{\deg} & sensor field of view \\
		\textit{H} & \si{\meter} & building width \\
		$Q*$ & \si{\watt\per\square\meter} & net radiation flux density \\
		$\Delta Q_A$ & \si{\watt\per\square\meter} & net advective heat flux density \\
		$\Delta Q_G$ & \si{\watt\per\square\meter} & net heat storage flux density \\
		$Q_E$ & \si{\watt\per\square\meter} & latent heat flux density \\
		$Q_F$ & \si{\watt\per\square\meter} & anthropogenic heat flux density \\
		$Q_H$ & \si{\watt\per\square\meter} & sensible heat flux density \\
		$t$ & \si{\minute} & time \\
		T\textsubscript{air} & \si{\kelvin} & air temperature \\
		T\textsubscript{surf} & \si{\kelvin} & surface temperature \\
		T\textsubscript{comp} & \si{\kelvin} & complete urban surface temperature\\
		T\textsubscript{plan} & \si{\kelvin} & urban surface temperature viewed from a sensor in the nadir\\
		T\textsubscript{hem} & \si{\kelvin} & urban surface temperature viewed from a hemispherical remote sensor\\
		aUHI & \si{\kelvin} & air temperature urban heat island effect \\
		clUHI & \si{\kelvin} & canopy layer air temperature urban heat island effect \\
		sUHI & \si{\kelvin} & surface temperature urban heat island effect \\
		$V_B$ & - & effects from background climate on observation of a given meteorological variable \\
		$V_H$ & - & effects from the urban environment on observation of a given meteorological variable \\
		$V_L$ & - & effects from local topography on observation of a given meteorological variable \\
		$V_M$ & - & observed meteorological variable \\
		$V_{M, U} $ & - & observed meteorological variable (urban) \\
		$V_{M  R}$ & - & observed meteorological variable (rural) \\
		$z$ & \si{\meter} & layer thickness (air, surface, or subsurface)\\
		\bottomrule
	\end{tabular} 
\end{table}

\captionsetup[table]{list=yes}

%\printglossaries
\cleardoublepage
\phantomsection		% allows hyperref to link to the correct page

% Change page numbering back to Arabic numerals and have it in the upper right corner
\pagenumbering{arabic}
\renewcommand{\headrulewidth}{0pt}
\pagestyle{fancy}
\fancyhf{}
\rhead{\thepage}
\makeatletter
\let\ps@plain\ps@fancy % plain style = fancy style
\makeatother

