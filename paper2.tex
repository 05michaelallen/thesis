\chapter{A climatology of urban surface heat islands derived from hemispherical radiometric surface temperatures}

\section{Introduction}

The temperature of the surface is integral in understanding, predicting, and modeling boundary-layer air temperature patterns, surface energy balances, and, in urban areas, has important implications for human thermal comfort and building energy usage. Urban modification of surface geometry and thermal, radiative, moisture, and aerodynamic properties results in differential surface heating and cooling patterns and strong microscale spatiotemporal variations in urban surface temperature. Integrated up to larger scales, urban areas tend to store and generate more heat relative to non-built surroundings and, as a result, are generally warmer --- a phenomenon termed the urban heat island effect (UHI). As such, accurate, spatiotemporally continuous and geometrically representative characterization of surface temperatures has long been a goal in urban climatology. The proliferation of satellite and aerial thermal infrared (TIR) remote sensing has enabled spatially-extensive, large-scale characterizations of surface climates at ever improving spatial and spectral resolutions. Such campaigns have elucidated urban surface temperature (T\textsubscript{surf}) and surface urban heat island (sUHI) patterns globally at large spatial scales. However, technological improvements in TIR remote sensing have yet to address a three potential sources of error when applied in urban areas: 

\begin{enumerate}
	\item Geometric undersampling of 3-dimensional terrain
	\item Temporal discontinuity in overpass cycles and sensor sampling regimes
	\item Clear-sky bias
\end{enumerate}

These biases present a potentially significant source of error by failing to capture micro-scale temporal and geometric variations in urban T\textsubscript{surf}. 

Inter-site comparison is the crux of UHI analysis, thus it is imperative that urban surface temperature measurements are accurate and representative of coherent urban patches. Meta-analysis of air temperature UHI literature shows that these goals are not often not satisfied \cite{Stewart2011}. Given the relative difficulty in retrieving accurate, representative urban surface temperatures, similar conclusions are likely for sUHI analysis. In spite of this fact, and the short period over which large-scale generalizable methods for urban surface temperature acquisition have been available, study of sUHI has expanded significantly in the last twenty years \cite{Peng2012,Voogt2003}. This paper presents a method to derive hemispherical radiometric urban T\textsubscript{surf} (Them) from hemispherical upwelling TIR as measured by inverted pyrgeometers. This method was developed to addresses biases inherent in traditional methods for urban T\textsubscript{surf} retrieval by providing temporally continuous, geometrically representative\footnote{Them is not perfectly representative of urban geometry. This is best illustrated by visualizing an urban area from the perspective of a downward facing ‘fish-eye’ camera. However, it is undoubtedly more representative of urban geometry than most 2-dimensional views of the surface. This fact is discussed in more depth in sectionXXXX.}  urban T\textsubscript{surf} under all-sky conditions for sUHI analysis. These measurements are often made as a part of the net radiation determination for urban energy balance studies.

\section{Bias in Thermal remote sensing}
Geometric biases in remote sensing of the urban surface are a result of its 3-dimensional, convoluted structure. Compared to flat terrain, complex urban surface geometry modifies receipt of incoming solar radiation and traps a portion of reflected solar and outgoing terrestrial radiation. Differential heating and shading patterns of vertical, sloped, and horizontal urban facets manifests in significant micro-scale contrasts in surface temperature. As such, measured radiometric surface temperature varies based on sensor field-of-view, viewing angle and direction, and sun-surface geometry – this directional dependence of urban surface temperature is termed ‘effective thermal anisotropy’ \cite{Voogt1998a}. Traditional satellite or airborne remote sensing platforms, by viewing the surface in the nadir, sample only a fraction of the complete urban surface and fail to capture this effect – leading to directional biases in measured urban surface temperature. Geometric undersampling by a remote sensor in the nadir manifests in an overestimation of daytime temperatures and underestimation of nighttime temperatures \cite{Adderley2015}. However, the magnitude and diurnal pattern of this bias is dependent on urban patch characteristics (canyon height-to-width ratio, canyon orientation and materials, vegetation coverage, etc.) and sensor viewing angle. Thus, parameterization schemes to account for urban effective anisotropy are difficult to generalize across urban sites, sensor types, and sensor orientations.

In addition to undersampling the urban surface, most thermal remote sensing platforms yield an instantaneous ‘snap-shot’ and cannot characterize temporally continuous T\textsubscript{surf} patterns. Temporal discontinuities in thermal remote sensing result in a number of sources of bias operating over a wide range of time scales. Aerial and satellite thermal remote sensing require clear sky conditions (clouds are opaque with respect to thermal infrared radiation). Hence, long term satellite characterizations of sUHI are biased towards conditions that maximize macro-scale urban-rural contrasts in surface temperature. This results in an overestimation of “all-sky” sUHI. Although

Over a day, sUHI is generally largest in the late afternoon, near solar noon, and just after sunset. Continuous sUHI analysis indicates that it exhibits a large diurnal amplitude. Satellite overpass cycles rarely coincide with sUHI maximums and are not standard across cities or platforms. Thus, analysis of sUHI patterns and magnitudes across cities and instrument platforms is difficult.  High-frequency fluctuations in surface temperature add an additional – and understudied – bias to thermal remote sensing of urban surface temperature. Time series analysis of urban surface temperature shows significant microscale (second to minute) fluctuations in temperature \cite{Christen2012}. Most thermal remote sensors observe surface temperatures instantaneously (rather than temporally averaged) and are potentially contaminated by microscale fluctuations. This is particularly salient in urban environments, where a large variety of fabric materials can produce significant directional contrasts in thermal admittance – and thus spatial variations in the magnitude of microscale fluctuations depending on the facet material types viewed by the sensor. The effect of this phenomenon on thermal remote sensing has not been extensively studied, however, the magnitude of microscale fluctuations in surface temperature is significant relative to a typical sUHI signal and thus constitutes a potentially large source of bias.

Both geometric and temporal shortcomings limit the representativity of traditional remote sensed evaluations of urban T\textsubscript{surf} and sUHI. The magnitude of these biases has not been extensively studied in climatological form. This paper presents a method to both assess the magnitude of and overcome these biases.