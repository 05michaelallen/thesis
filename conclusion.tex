\begin{bibunit}

\rhead{\thepage}

\chapter{Conclusion}

This study presents the first long-term, time-continuous analysis of urban T\textsubscript{surf} and sUHI that incorporates representative sampling of the three-dimensional urban surface. It is based on hemispheric upwelling longwave radiation measurements that are commonly made as a part of urban energy balance assessment and constitute a hitherto untapped resource for urban T\textsubscript{surf} and sUHI analysis. The analysis contributes to a more complete understanding of the temporal and geometric nature of the urban effect on land T\textsubscript{surf} and sUHI. 

\section{Summary of results}

Chapter \ref{paper1} details a hemispherical atmospheric correction method that uses version 4.1 of the MODTRAN radiative transfer code \citep{Berk1987} in conjunction with the Surface-Sensor-Sun Urban Model \citep{Soux2004} to derive hemispherical radiometric surface temperatures from longwave radiation upwelling from the urban surface measured from a downward facing pyrgeometer. The correction method is necessary for two reasons:

\begin{enumerate}
	\item To retrieve T\textsubscript{surf} that is representative of the temperature at which the surface is radiating (known as radiometric T\textsubscript{surf}) by removing effects from absorption and emission by the atmosphere in between the surface and the sensor.
	\item To facilitate intersite analysis of T\textsubscript{surf}, where different ambient conditions, surface-sensor-sun geometries, and instrument types can have differential effects remote sensed TIR.
\end{enumerate}

 The correction method was developed to be readily adaptable to other sensor types and study sites by accounting for complex three-dimensional surface geometry and non uniform spectral sensor response. 

Analysis of an eight months of corrected T\textsubscript{hem, r} shows that atmospheric effects on inferred T\textsubscript{hem} are large relative to typical sUHI magnitudes and show significant diurnal and seasonal variation. Relationships between correction magnitude and meteorological variables are complex. Correction magnitudes show strong positive relationships with $\Delta$T\textsubscript{surf-air} and incoming solar radiation. Conversely, although water vapor is the main absorber of broadband TIR radiation, it does not exert significant influence on correction magnitudes, as atmospheric transmittance shows little variation at typical humidities. To facilitate application of the method, a parameterization was developed incorporating a small number of generalizable radiative transfer simulations (included in Appendix \ref{appendixa}) and information about site geometry to retrieve radiometric hemispherical T\textsubscript{surf} en masse using measurements of upwelling longwave, humidity, and T\textsubscript{air}. 

In addition, Chapter \ref{paper1} explores the effect of surface geometry on remote sensed urban T\textsubscript{surf} for three representations of the urban surface (complete, plan, and hemispherical). Analysis shows significant overestimation of T\textsubscript{comp} by T\textsubscript{plan} during the day and underestimation at night, particularly under clear sky conditions. By day, T\textsubscript{hem} overestimates T\textsubscript{comp}, but to a lesser degree than T\textsubscript{plan}. Daytime overestimation of T\textsubscript{comp} by T\textsubscript{hem} is more consistent under a wide variety of synoptic conditions, whereas overestimation by T\textsubscript{plan} fluctuates significantly, resulting in poor performance under clear-sky "satellite friendly" conditions. As measurements from narrow-FOV sensors in the nadir make up the bulk of the satellite urban T\textsubscript{surf} record, our understanding of urban T\textsubscript{surf} likely includes a significant warm bias. These biases are reflected in analysis of the effect of sensor-surface geometry on sUHI in Chapter \ref{paper2}.

Chapter \ref{paper2} includes an analysis of two long term, time-continuous climatologies of hemispherical sUHI, one in Basel, Switzerland and another in Vancouver, Canada. sUHI for both study sites shows significant diurnal, day-to-day, and seasonal variability. sUHI is highly dependent on meteorological conditions, particularly accumulated solar radiation over a day and wind velocity. However, as sUHI is the result of differences in heating and cooling rates between the city and its non urban surroundings, its relationship to meteorological controls is complex. Analysis of sUHI development under a range of mid-latitude synoptic conditions lead to the development of the first observationally-supported, generalizable model for sUHI development presented in Section \ref{gen}.

Analysis of the effect of sensor-surface geometry on sUHI is also included in Chapter \ref{paper2}. Results are similar to those found in Chapter \ref{paper1}, with significant daytime overestimation of sUHI\textsubscript{comp} by sUHI\textsubscript{plan}, particularly during hot clear sky conditions. Directional biases in remote sensed assessment of sUHI change observed magnitudes and the diurnal character of sUHI by modifying patterns of observed sUHI in the hours near sunset. Thus, satellite based assessments of the diurnal pattern of sUHI may lead to misrepresentations of its 'true' diurnal character. 

\section{Limitations and future work}

%The convoluted and complex nature of the urban surface makes remote sensing urban T\textsubscript{surf} difficult. Although the method detailed in this study provides T\textsubscript{surf} that is time continuous and more representative of the complete urban T\textsubscript{surf} than conventional remote sensing methods - particularly under clear sky conditions - it is not without some limitations. 

The complex nature of the urban surface introduces myriad shortcomings and biases into conventional remote sensing of urban thermal climates - this method is no exception. Sensor placement sensitivity tests in Section \ref{placement} corroborate findings in \citet{Roberts2010,Adderley2015} and indicate that longwave radiation upwelling from the urban surface as measured from a downward-facing radiometer is spatially variant when measured from heights below approximately three to five times mean building height. Most pyrgeometers (including those used in this study) are mounted below this height and are subject to a bias towards facets that make up a disproportionately large fraction of the sensor FOV. For sensors mounted at less than optimal heights, the effect of this bias on derived T\textsubscript{hem, r} is largely dependent on sensor placement relative to rooftop facets, with better performance when mounted offset from roof areas. Pyrgeometers at the Sperrstrasse and Sunset tower sites are placed far enough from rooftops to render this bias relatively small. However, future studies should strive for pyrgeometer heights that are more optimal not only to yield more geometrically representative T\textsubscript{hem, r}, but also to ensure irradiances are representative of the complete urban longwave emission in energy balance assessments.

As the method developed in this study uses upwelling longwave measurements from fixed tower sites, it is not practical for large scale spatially-continuous and spatially extensive analysis of sUHI. This limits the method to mid- and low-rise neighborhoods. However, by combining concepts from other work, the method can be adapted to retrieve spatiotemporally continuous urban T\textsubscript{hem, r} from mobile measurements of upwelling longwave irradiances. \citet{Sugawara2006} used helicopter-mounted pyrgeometers and narrow-FOV infrared thermometers in aerial traverses to successfully quantify effects from urban effective anisotropy on upwelling longwave radiation, but did not incorporate a robust atmospheric correction routine to account for spatially variant surface geometry to remove atmospheric effects or retrieve T\textsubscript{hem, r}. Determining the distance to the surface for a moving hemispherical sensor viewing a complex, three-dimensional surface is difficult. Ray-tracing methods similar to those used in \citet{Ceamanos2017} could provide a solution, but must be modified to return path lengths in addition to seen and unseen points for a mobile aerial sensor. In addition, sensitivity to ambient conditions (particularly the T\textsubscript{air} profile) will make accurate atmospheric correction computationally expensive in mobile applications of the method. Notwithstanding evaluations of method performance in such applications. If these problems are solved, application of the method in aerial traverses may be useful in quantifying the spatiotemporal character of urban effective anisotropy on remote sensed urban T\textsubscript{surf} as hemispherical instrumentation can be run simultaneously with near-ground and satellite based narrow-FOV sensors. Such analysis could be used to develop parameterizations to modify satellite retrieved urban T\textsubscript{surf} to better represent the complete urban T\textsubscript{surf}.

While satellite assessment of sUHI and urban T\textsubscript{surf} has been essential in elucidating the spatial character of urban effects on T\textsubscript{surf}, several of the questions posed nearly 30 years ago in \citet{Roth1989} remain unanswered. This should prompt not only critical assessment of the urban T\textsubscript{surf} record, but also the development and use of novel methods to fill knowledge gaps, quantify biases, and uncover methodological shortcomings. This method, and in particular its parameterization, is intended to aid in such an endeavor as it is readily adapted to other study sites and requires few meteorological inputs. According to the Urban Flux Network (\href{http://fluxnet.fluxdata.org/}{Urbanflux}) currently there are 26 operational flux towers in urban areas instrumented for assessment of the net radiation budget. In addition, many other datasets of urban energy balances from inactive and decommissioned towers and from campaigns not indexed by Urbanflux are available for analysis via this method. These datasets span several continents and climatic zones, and characterize energy balances for cities over a range of developing and developed nations, with many operational over very long (5+ years) time scales. Thus, the application of this method over existing untapped datasets, and as a component of future energy balance assessments, can not only could provide a wealth of novel urban T\textsubscript{surf} and sUHI analyses for urban climatology and could also can provide for analysis outside the traditional purview of urban climatology. The character of the urban effect on land T\textsubscript{surf} (and climate at large) has complex, multifaceted relationships with the layout, size, population density, and development regimes of a given city as well as factors that are more difficult to observe, including social structure and culture. Long term, time continuous analysis of urban T\textsubscript{surf} and sUHI can help to better understand these links. 

\section{Final remarks}

%Broadly, the method, its parameterization, and the analysis presented in this study not only provides for novel analysis of the urban effect on T\textsubscript{surf}, but also aims to prompt further critical assessment of the urban T\textsubscript{surf} and the sUHI record. 

Broadly, the method, its parameterization, and the analysis presented in this study were conducted to provide for novel analyses of the urban effect on T\textsubscript{surf} and to prompt further critical assessment of the urban T\textsubscript{surf} and sUHI record. In addition, this method was developed in order to contribute to addressing methodological and conceptual questions outlined in \citet{Roth1989} by filling knowledge gaps in the urban T\textsubscript{surf} and sUHI record. These analyses help to understand the temporal and geometric character of urban T\textsubscript{surf} and to facilitate more effective and representative study of the urban effect on T\textsubscript{surf}. Indeed, if similar conceptual flaws and shortcomings to those found for the aUHI in \citet{Stewart2011} are likely in sUHI analysis, quantification of instrument and methodological biases and shortcomings in remote sensing of urban T\textsubscript{surf} and the introduction of new methods to overcome these shortcomings may bring us closer to uncovering the true geometric, temporal, and spatial effects of cities on land surface temperature and on the climate at large.

\cleardoublepage 
\phantomsection  
\renewcommand*{\bibname}{References}
\addcontentsline{toc}{section}{\textbf{References}}


\putbib
\end{bibunit}