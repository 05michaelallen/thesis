\begin{bibunit}

\rhead{\thepage}

\chapter{Concluding statements}

This study presents the first long-term, time-continuous analysis of urban T\textsubscript{surf} to contribute to a more complete understanding of the temporal and geometric nature of the urban effect on land T\textsubscript{surf} and sUHI. 

Chapter \ref{paper1} details a hemispherical atmospheric correction method that uses version 4.1 of the MODTRAN radiative transfer code \citep{Berk1987} in conjunction with the Surface-Sensor-Sun Urban Model \citep{Soux2004} to derive atmospherically corrected hemispherical radiometric surface temperatures from upwelling longwave radiation measured from a downward facing pyrgeometer. These temperatures are representative of the temperature at which the urban surface is radiating. The correction method was developed to be readily adaptable to other sensor types and study sites by accounting for surface geometry and non uniform spectral sensor response. These measurements are commonly made as a part of urban energy balance assessment and constitute a hitherto untapped resource for urban T\textsubscript{surf} and sUHI analysis. Atmospheric effects on inferred T\textsubscript{hem} are significant relative to typical sUHI magnitudes and show significant diurnal and seasonal variation. Relationships between correction magnitude and meteorological variables are complex. Although water vapor is the main absorber of broadband TIR radiation, it does not exert significant influence on correction magnitudes, as atmospheric transmittance shows little variation at typical humidities. Correction magnitudes show strong positive relationships with to $\Delta$T\textsubscript{surf-air} and incoming solar radiation. To further facilitate method usability, a parameterization was developed incorporating a small number of generalizable radiative transfer simulations (included in \ref{appendixa}) and information about site geometry to retrieve radiometric hemispherical T\textsubscript{surf} from measurements of upwelling longwave en masse. 

In addition, Chapter \ref{paper1} explores the effect of surface geometry on remote sensed urban T\textsubscript{surf} for three representations of the urban surface (complete, plan, and hemispherical). Analysis shows significant overestimation (underestimation) of T\textsubscript{comp} by T\textsubscript{plan} by day (night), particularly under clear sky conditions. By day, T\textsubscript{hem} overestimates T\textsubscript{comp}, but to a lesser degree than T\textsubscript{plan}. Daytime overestimation of T\textsubscript{comp} by T\textsubscript{hem} is more consistent under a wide variety of synoptic conditions, whereas overestimation by T\textsubscript{plan} fluctuates significantly, resulting in poor performance under clear-sky "satellite friendly" conditions. As these measures make up the bulk of the satellite urban T\textsubscript{surf} record, our understanding of urban T\textsubscript{surf} likely includes a significant warming bias. These biases are reflected in sUHI analysis in \ref{paper2}.

Chapter \ref{paper2} includes an analysis of two long term, time continuous climatologies of hemispherical sUHI, one in Basel, Switzerland and another in Vancouver, Canada. sUHI for both study sites shows significant diurnal, day-to-day, and seasonal variability. sUHI is highly dependent on meteorological conditions, particularly accumulated solar radiation over a day and wind velocity. However, as sUHI is the result of differences in heating and cooling rates between the city and its non urban surroundings, its relationship to meteorological controls is complex. Analysis of sUHI development under a range of mid-latitude synoptic conditions lead to the development of a data supported generalizable model for summertime sUHI development presented in section \ref{gen}.

Analysis of the directional dependence of sUHI is also included in Chapter \ref{paper2}. Similar results are observed as those found in Chapter \ref{paper1}, with significant overestimation of sUHI\textsubscript{comp} by T\textsubscript{plan}, particularly during hot clear sky conditions. Not only did directional biases in remote sensed sUHI change observed magnitudes, these biases also changed the diurnal character of sUHI, informing patterns of sUHI in the hours during near sunset. Failure, in satelite based assessment of the diurnal pattern of sUHI may lead to misrepresentations of its 'true' diurnal character. 

\section{Final remarks}

Broadly, the method, parameterization, and analysis presented in this study not only provides for novel analysis of the urban effect on T\textsubscript{surf}, but also aims to prompt further critical assessment of the urban T\textsubscript{surf} and sUHI record. Indeed, if similar conceptual flaws and shortcomings to those found for the aUHI in \citet{Stewart2011} are likely in sUHI analysis, quantification of instrument and methodological biases and shortcomings in remote sensing of urban T\textsubscript{surf} and the introduction of new methods to overcome these shortcomings may bring us closer to and to addressing the warnings raised in \citet{Stewart2011}.

\cleardoublepage 
\phantomsection  
\renewcommand*{\bibname}{References}
\addcontentsline{toc}{section}{\textbf{References}}


\putbib
\end{bibunit}