\begin{bibunit}

\rhead{\thepage}

\chapter{Concluding statements}

This study presents the first long-term, time-continuous analysis of urban T\textsubscript{surf} to contribute to a more complete understanding of the temporal and geometric nature of the urban effect on land T\textsubscript{surf} and sUHI. 

Chapter \ref{paper1} details a hemispherical atmospheric correction method that uses version 4.1 of the MODTRAN radiative transfer code \citep{Berk1987} in conjunction with the Surface-Sensor-Sun Urban Model \citep{Soux2004} to derive atmospherically correct hemispherical radiometric surface temperatures from upwelling longwave radiation measured from a downward facing pyrgeometer. The correction method was developed to be readily adaptable to other sensor types and study sites, these measurements are commonly made as a part of urban energy balance assessment. By accounting for surface geometry and spectrally non-uniform sensor response, the correction method is readily adaptable to other sensor types and study sites. Atmospheric effects on inferred T\textsubscript{hem} are significant relative to typical sUHI magnitudes and show significant diurnal and seasonal variation. Correction magnitudes are particularly sensitive to $\Delta$T\textsubscript{surf-air}. Although water vapor is the main absorber of broadband TIR radiation, it does not exert significant influence on correction magnitudes, as atmospheric transmittance shows little variation at typical humidities. 

In addition, Chapter \ref{paper1} explores the effect of surface geometry on remote sensed urban T\textsubscript{surf} for three representations of the urban surface (complete, plan, and hemispherical). Analysis shows significant overestimation (underestimation) of T\textsubscript{comp} by T\textsubscript{plan} by day (night), particularly under clear sky conditions. By day, T\textsubscript{hem} overestimates T\textsubscript{comp}, but to a lesser degree than T\textsubscript{plan}. Daytime overestimation of T\textsubscript{comp} by T\textsubscript{hem} is more consistent under a wide variety of synoptic conditions, whereas overestimation by T\textsubscript{plan} fluctuates significantly, resulting in poor performance under clear-sky "satellite friendly" conditions. As make up the bulk of the satellite urban T\textsubscript{surf} record, our understanding of urban T\textsubscript{surf} likely includes a significant warming bias. These biases are reflected in sUHI analysis in \ref{paper2}.

Analysis of an eight month, time continuous climatology of hemispherical sUHI is included in Chapter \ref{paper2}.

\cleardoublepage 
\phantomsection  
\renewcommand*{\bibname}{References}
\addcontentsline{toc}{section}{\textbf{References}}


\putbib
\end{bibunit}